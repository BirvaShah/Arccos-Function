\documentclass{article}
\usepackage[utf8]{inputenc}
\usepackage[margin=0.1in]{geometry}
\usepackage{enumitem}
\title{Function:- arccos(x)
\begin{large}
\\
\Large Problem-2
\end{large}
}
\author{Birva Shah (Student ID: 40070973) }
\date{12 July 2019}

\usepackage{natbib}
\usepackage{graphicx}
\renewcommand{\baselinestretch}{1.6}
\begin{document}
\maketitle
\section{Assumptions}
\begin{itemize}
    \item In function $arccos(x)$, x is a real number.
    \item Function returns the value of $arccos(x)$ in radian. 
    \item If the argument of the function is NaN, then the result is NaN.
\end{itemize}  
\section{Requirements}
\begin{enumerate}[label=(\alph*)]
\item \textbf{ID = } REQ-1
\\
\begin{Large}
 \large \textbf{Type = } Functional Requirement
 \\
 \large \textbf{Version = } 1.0
 \\
 \large \textbf{Difficulty =} Easy
 \\
 \large \textbf{ Description  = }User shall give input value x between -1 and 1 inclusive to satisfythe constraint that the domain of the function $arccos(x)$ is $-1\le x\le1$.
 \\
  \large \textbf{Rationale =} The rationale behind this requirement is that the output of the function $arccos(x)$ is undefined if the value of x is not between -1 and 1 inclusive.
 \\
 
\end{Large}

\item \textbf{ID = } REQ-2
\\
\begin{Large}
 \large \textbf{Type = } Functional Requirement
 \\
 \large \textbf{Version = } 1.0
 \\
  \large \textbf{Difficulty =} Nominal
  \\
 \large \textbf{ Description  = } The function shall take input x to give the output of the function in radian. For example: $arccos(0.5)=1.4719...$.
 \\
  \large \textbf{Rationale =} The rationale behind this requirement is that only one input x which is real number is required to calculate result of arccos(x).
 \\

\end{Large}
\\

\item \textbf{ID = } REQ-3
\\
\begin{Large}
 \large \textbf{Type = } Functional Requirement
 \\
 \large \textbf{Version = } 1.0
 \\
  \large \textbf{Difficulty =} Nominal
  \\
 \large \textbf{ Description  = } The function shall calculate the value of $arccos(x)$ up to the precision of four decimals to get the stable output. For example: $arccos(0.5)=1.4719$
 \\
  \large \textbf{Rationale =} The rationale behind this requirement is that the function might give an output that has infinite number of decimals points. 
 \\

\end{Large}
\end{enumerate}
\begin{thebibliography}{9}
\bibitem{rapidtableswebsite}
RapidTables,
\\\texttt{https://www.rapidtables.com/math/trigonometry/arccos.html}
\bibitem{emathhelp}
Emathhelp,
\\\texttt{https://www.emathhelp.net/notes/algebra-2/trigonometry/function-y-arccos-x/}
\bibitem{microsoft}
Microsoft,
\\\texttt{https://docs.microsoft.com/en-us/powerapps/maker/canvas-apps/functions/function-trig}
\bibitem{mathonweb}
Mathonweb,
\\\texttt{http://mathonweb.com/help_ebook/html/algorithms.htm#arcsin}
\bibitem{cliffsnotes}
CliffsNotes,
\\\texttt{https://www.cliffsnotes.com/study-guides/trigonometry/inverse-functions-and-equations/inverse-cosine-and-inverse-sine}
\end{thebibliography}
\newpage

\end{document}
