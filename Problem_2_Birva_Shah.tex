\documentclass{article}
\usepackage[utf8]{inputenc}
\usepackage[margin=0.1in]{geometry}
\title{Function:- arccos(x)
\begin{large}
\\
\Large Problem-2
\end{large}
}
\author{Birva Shah (Student ID: 40070973) }
\date{12 July 2019}

\usepackage{natbib}
\usepackage{graphicx}

\begin{document}
\maketitle
\section{Assumptions}
\begin{itemize}
    \item In function $arccos(x)$, x is a real number.
    \item Function returns the value of $arccos(x)$ in radians. 
    \item If the argument of the function is NaN, then the result is NaN.
\end{itemize}  
\section{Requirements}
\begin{itemize}
\item \textbf{ID \hspace{1.95cm}: } REQ-1
\\
\begin{Large}
 \large \textbf{Type \hspace{1.15cm} : } Functional Requirement
 \\
 \large \textbf{ Description  : }User input should be between -1 and 1 inclusive because the domain of the function $arccos(x)$ is $-1\le x\le1$.
 \\
  \large \textbf{Rationale \hspace{0.4cm}:} The rationale behind this requirement is that the output of the function $arccos(x)$ is undefined if the value of x is not between -1 and 1 inclusive.
 \\
 \large \textbf{Difficulty \hspace{0.45cm}:} Easy
\end{Large}

\item \textbf{ID \hspace{1.95cm}: } REQ-2
\\
\begin{Large}
 \large \textbf{Type \hspace{1.15cm} : } Functional Requirement
 \\
 \large \textbf{ Description  : } The output of the function is assumed to be in radians. One more function should be made to to convert radian values to degree values. 
 \\
  \large \textbf{Rationale \hspace{0.4cm}:} The rationale behind this requirement is that the scientific calculator gives the result in degree values.
 \\
 \large \textbf{Difficulty \hspace{0.45cm}:} Nominal
\end{Large}
\\

\item \textbf{ID \hspace{1.95cm}: } REQ-3
\\
\begin{Large}
 \large \textbf{Type \hspace{1.15cm} : } Functional Requirement
 \\
 \large \textbf{ Description  : } To define the inverse function, each value in the domain must correspond to exactly one value in the range and vice versa. 
 \\
  \large \textbf{Rationale \hspace{0.4cm}:} The rationale behind this requirement is that the original function i.e. $cos(x) where -1\le x\le1$ is one-to-one function. 
 \\
 \large \textbf{Difficulty \hspace{0.45cm}:} Nominal
\end{Large}
\end{itemize}
\begin{thebibliography}{9}
\bibitem{rapidtableswebsite}
RapidTables,
\\\texttt{https://www.rapidtables.com/math/trigonometry/arccos.html}
\bibitem{emathhelp}
Emathhelp,
\\\texttt{https://www.emathhelp.net/notes/algebra-2/trigonometry/function-y-arccos-x/}
\bibitem{microsoft}
Microsoft,
\\\texttt{https://docs.microsoft.com/en-us/powerapps/maker/canvas-apps/functions/function-trig}
\bibitem{mathonweb}
Mathonweb,
\\\texttt{http://mathonweb.com/help_ebook/html/algorithms.htm#arcsin}
\bibitem{cliffsnotes}
CliffsNotes,
\\\texttt{https://www.cliffsnotes.com/study-guides/trigonometry/inverse-functions-and-equations/inverse-cosine-and-inverse-sine}
\end{thebibliography}
\end{document}
