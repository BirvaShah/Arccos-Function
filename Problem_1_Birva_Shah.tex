\documentclass[12pt]{article}
\usepackage{graphicx}
\usepackage[legalpaper, a4paper,margin=1mm]{geometry}
\usepackage{textcomp}
\usepackage{environ}
\NewEnviron{myequation}{%
    \begin{equation}
    \scalebox{1.2}{$\BODY$}
    \end{equation}
    }
\graphicspath{ {../images/} }
\title{Function:- arccos(x)}
\author{Birva Shah (Student ID: 40070973)}
\begin{document}
\maketitle
\section{Problem-1}
\subsection{Definition}
The arccosine of x is defined as the inverse cosine function of x when -1$\leq$x$\leq$1.
When the cosine of y is equal to x:
\begin{myequation}%
$$\cos y = x$$
\end{myequation}
Then the arccosine of x is equal to the inverse cosine function of x, which is equal to y:
\begin{myequation}%
$$\arccos(x) = \cos^{-1} x = y$$
\end{myequation}
\subsection{Domain and Range}
The domain of arccos(x) is -1$\leq$x$\leq$1 and the range of arccos(x) is 0$\leq$y$\leq$$\pi$ \big(0\textdegree$\leq$y$\leq$180\textdegree\big).
\subsection{Characteristics of arccos(x)}
\begin{itemize}
  \item This function is neither even nor odd.
  \item It is a decreasing function.
  \item Graph of arccos(x)
\end{itemize}
\includegraphics[width=100mm,scale=0.5]{arccos}
\begin{thebibliography}{9}
\bibitem{rapidtableswebsite}
RapidTables,
\\\texttt{https://www.rapidtables.com/math/trigonometry/arccos.html}
\bibitem{emathhelp}
Emathhelp,
\\\texttt{https://www.emathhelp.net/notes/algebra-2/trigonometry/function-y-arccos-x/}
\end{thebibliography}
\end{document}