\documentclass[final]{beamer}

\usepackage[scale=1.24]{beamerposter} 


\usetheme[faculty=chemo]{fibeamer}

\newlength{\sepwid}
\newlength{\onecolwid}
\newlength{\twocolwid}
\newlength{\threecolwid}
\setlength{\paperwidth}{46.8in} % A0 width: 46.8in
\setlength{\paperheight}{33.1in} % A0 height: 33.1in
\setlength{\sepwid}{0.024\paperwidth} % Separation width (white space) between columns
\setlength{\onecolwid}{0.21\paperwidth} % Width of one column
\setlength{\twocolwid}{0.451\paperwidth} % Width of two columns
\setlength{\threecolwid}{0.678\paperwidth} % Width of three columns
%\setlength{\topmargin}{-0.5in} % Reduce the top margin size
%-----------------------------------------------------------

\usepackage{graphicx}  % Required for including images

\usepackage{booktabs} % Top and bottom rules for tables

%----------------------------------------------------------------------------------------
%	TITLE SECTION 
%----------------------------------------------------------------------------------------

\title{SOEN-6011 Project Poster\\
\LARGE F1: $arccos(x)$
}
\author{Birva Shah} % Author(s)

\institute{Concordia University, Montreal} % Institution(s)

%----------------------------------------------------------------------------------------

\begin{document}
\addtobeamertemplate{block end}{}{\vspace*{2ex}} % White space under blocks
\addtobeamertemplate{block example end}{}{\vspace*{2ex}} % White space under example blocks
\addtobeamertemplate{block alerted end}{}{\vspace*{2ex}} % White space under highlighted (alert) blocks

\setlength{\belowcaptionskip}{2ex} % White space under figures
\setlength\belowdisplayshortskip{2ex} % White space under equations
%\begin{darkframes} % Uncomment for dark theme, don't forget to \end{darkframes}
\begin{frame} % The whole poster is enclosed in one beamer frame

%==========================Begin Head===============================

  \begin{columns}
   \begin{column}{\linewidth}
    \vskip1cm
    \centering
    \usebeamercolor{title in headline}{\color{fg}\Huge{\textbf{\inserttitle}}\\[0.5ex]}
    \usebeamercolor{author in headline}{\color{fg}\Large{\insertauthor}\\[1ex]}
    \usebeamercolor{institute in headline}{\color{fg}\large{\insertinstitute}\\[1ex]}
    \vskip1cm
   \end{column}
   \vspace{1cm}
  \end{columns}
 \vspace{1cm}

%==========================End Head===============================

\begin{columns}[t] % The whole poster consists of three major columns, the second of which is split into two columns twice - the [t] option aligns each column's content to the top

\begin{column}{\sepwid}\end{column} % Empty spacer column

\begin{column}{\onecolwid} % The first column

%----------------------------------------------------------------------------------------
%	OBJECTIVES
%----------------------------------------------------------------------------------------

\begin{exampleblock}{Definition}

The arccosine of x is defined as the inverse cosine function of x when -1$\leq$x$\leq$1.
When the cosine of y is equal to x:
\begin{myequation}%
$$\cos y = x$$
\end{myequation}
Then the arccosine of x is equal to the inverse cosine function of x, which is equal to y:
\begin{myequation}%
$$\arccos(x) = \cos^{-1} x = y$$
\end{myequation}
\end{exampleblock}
\begin{exampleblock}{Domain and Range}
The domain of arccos(x) is -1$\leq$x$\leq$1 and the range of arccos(x) is 0$\leq$y$\leq \pi$ \big(0\textdegree$\leq$y$\leq$180\textdegree\big).
\end{exampleblock} 
\begin{exampleblock}{Characteristics of arccos(x)}
\begin{itemize}
  \item This function is neither even nor odd.
  \item It is a decreasing function.
  \item Graph of arccos(x)
  \includegraphics[width=4cm]{arccos}
\end{itemize}
\end{exampleblock}

\end{column} % End of the first column

\begin{column}{\sepwid}\end{column} % Empty spacer column

\begin{column}{\twocolwid} % Begin a column which is two columns wide (column 2)

\begin{columns}[t,totalwidth=\twocolwid] % Split up the two columns wide column

\begin{column}{\onecolwid}\vspace{-.74in} % The first column within column 2 (column 2.1)

\end{column} % End of column 2.1
\begin{column}{\sepwid}\end{column} % Empty spacer column

\begin{column}{\onecolwid}\vspace{-.74in} % The second column within column 2 (column 2.2)

\end{column} % End of column 2.2

\end{columns} 
\begin{columns}[t,totalwidth=\twocolwid] % Split up the two columns wide column again

\begin{column}{\onecolwid} % The first column within column 2 (column 2.1)


\end{column} % End of column 2.1
\begin{column}{\sepwid}\end{column} % Empty spacer column

\begin{column}{\onecolwid} % The second column within column 2 (column 2.2)

\end{column} % End of column 2.2

\end{columns} % End of the split of column 2

\end{column} % End of the second column

\begin{column}{\sepwid}\end{column} % Empty spacer column

\begin{column}{\onecolwid} % The third column


\begin{tabular}{rr}
\hspace{0.3\linewidth} & \includegraphics[width=0.5\linewidth]{img/logo.png}
\end{tabular}

%----------------------------------------------------------------------------------------

\end{column} % End of the third column

\begin{column}{\sepwid}\end{column} % Empty spacer column

\end{columns} % End of all the columns in the poster

\end{frame} % End of the enclosing frame
%\end{darkframes} % Uncomment for dark theme
\end{document}