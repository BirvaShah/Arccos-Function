% Generated by GrindEQ Word-to-LaTeX 
\documentclass[10pt,letterpaper]{article} 
\usepackage{graphicx} %package to manage images
\graphicspath{ {./images/} }
\usepackage{textcomp}
\usepackage[margin=0.5in]{geometry}
\usepackage[english]{babel}
\usepackage[utf8]{inputenc}

\usepackage{biblatex} %Imports biblatex package
\addbibresource{sample.bib}
\title{Function F1: Arccos(x)
\begin{large}
\\\Large SOEN-6011 Project\\
\LARGE Deliverable- 3
\\\Large Problem- 5
\\\Large Code Review Report of Function F2: $tan(x)$
\end{large}
}
\author{Birva Shah (Student ID: 40070973) }
\date{2 August 2019}

\begin{document}
\maketitle
\textbf{Github link:} https://github.com/BirvaShah/SOEN6011\_Project
\\
\textbf{Function F2 reviewed from Github link:} https://github.com/Hetvishah1710/tanx 
\section{Code Review}
Code review which is also called peer review is a phase in the software development process in which the programmers and peer reviewers come together to review a code. Their main aim is to find potential errors, check consistency of program design and to check adherence to coding standards and proper comments. This phase is relatively inexpensive rather than the more expensive process of handling, locating, and fixing bugs during later stages of development or after programs are delivered to users. There are various approaches and tools available for reviewing a source code, for example, PMD(Eclipse IDE plugin), Gerrit, Github, SpotBugs(Eclipse IDE Plugin), etc. We can also review a code manually. 
\section{Manual Code Review}
I have reviewed F2 function source code manually as well as using tools. In manual code review, I have checked the coding convention styles that we as a team had decided to follow. 
\begin{itemize}
\item The source code is compiling successfully.
\item I have checked the input and output of the code for verification. The author has put an effort to use a Memento pattern to enhance the design of the function. 
\item Error handling is properly done for values outside the domain as shown in Figure 1. However, if the user input is not a number then the program throws un-handled exceptions as an error. I would suggest the author to use exception handling to handle NaN inputs.
\begin{center}
    \includegraphics[width=0.50\textwidth]{images/2.PNG}\\
  \caption{Figure 1}
\end{center} 
\item The author has followed the coding styles particularly naming conventions of classes and functions as we had decided in a team. However, some variable names are found to be not adhering the purpose.
\item From Figure 2, I would suggest the author to differentiate the main output and the output of supporting functions. Our calculator aims to give the clear output of the desired function only. Moreover, it is suggested that before committing the code, author should check the spelling/grammar errors seen in the output. So that the user do not misunderstand or misinterpret the output.
\begin{center}
    \includegraphics[width=0.50\textwidth]{images/1.PNG}\\
  \caption{Figure 2}
\end{center} 
\end{itemize}
\section{Automated Code review using Tools}
Automated code review software checks source code for compliance with a predefined set of rules or best practices. The use of analytical methods to inspect and review source code to detect bugs has been a standard development practice. This process can be accomplished both manually and in an automated fashion. With automation, software tools provide assistance with the code review and inspection process. The review program or tool typically displays a list of warnings (violations of programming standards). \cite{Wikipedia}
\subsection{Tools used to conduct code review}
\subsubsection{PMD:}
PMD stands for Programming Mistake Detector.It is a free source code analysis tool which helps us to find the bugs in our java code and improve the code quality. I have used Eclipse PMD plugin to review the code.
\begin{itemize}
    \item Figure 3 shows the number of violations detected in the source code of Function $tan(x)$. The number of violations are more because of the console (textual) output of the function. The function uses \ttextbf{SystemPrintln} rule to print the output which is considered as violation in any automated tool. However, we can ignore/remove such type of violations as the author may have chose to make a console application according to Project Description. Other violations such as \ttextbf{DataFlowAnomalyAnalysis, LocalVariableCouldBeFinal} can be removed from violations by Right-clicking on that violation in Violations Outline in PMD and disabling the rule. Such violations are important only in industrial based large projects, I have disabled above three unnecessary violations for this code review purpose.
\begin{center}
    \includegraphics[width=1.0\textwidth]{images/4.PNG}\\
  \caption{Figure 3}
\end{center} 
    \item Figure 4 shows some of the warning violations. 
    \begin{center}
    \includegraphics[width=1.0\textwidth]{images/5.PNG}\\
  \caption{Figure 4}
\end{center} 
\item Figure 5 shows some of the important violations which can be fixed first to avoid future conflicts and violations.  
    \begin{center}
    \includegraphics[width=1.0\textwidth]{images/6.PNG}\\
  \caption{Figure 5}
\end{center}
\item Figure 6 shows some of the urgent violations for which the actions can be taken at high priority.  
    \begin{center}
    \includegraphics[width=1.0\textwidth]{images/8.PNG}\\
  \caption{Figure 6: Tan.java}
\end{center}
\item Figure 7 shows critical violations with error messages. Author can consider looking into it as it may result in code failure.
    \begin{center}
    \includegraphics[width=1.0\textwidth]{images/9.PNG}\\
  \caption{Figure 7}
\end{center}
\item There are no \textbf{blocker violations} in code.
\end{itemize}
\subsubsection{Checkstyle:}
Checkstyle is a development tool to help programmers write Java code that adheres to a coding standard. It automates the process of checking Java code to reduce the task of programmers. This makes it ideal for projects that want to enforce a coding standard. \cite{Sourceforge}
\\\\
I have ran the checkstyle plugin from Eclipse IDE to check the adherence of coding standards maintained in the source code. As shown in figure 8, there are 0 violations found. \textbf{This means that the author has taken care of writing a code by following the best practices and coding standards as defined in Checkstyle.}  
\begin{center}
    \includegraphics[width=1.0\textwidth]{images/10.PNG}\\
  \caption{Figure 8}
\end{center}

\begin{thebibliography}{9}
\bibitem{Search Software Quality}
Search Software Quality,\\
\url{https://searchsoftwarequality.techtarget.com/definition/code-review}
\bibitem{Wikipedia}
Wikipedia,\\
\url{https://en.wikipedia.org/wiki/Automated_code_review}
\bibitem{Sourceforge}
Sourceforge,\\
\url{https://checkstyle.sourceforge.io/}
\end{thebibliography}
\end{document}
